\chapter{Conditional Expressions}
\section{If Statements}
An \verb!if! statement is used to execute a section of code if an expression 
results to \verb!true!. A great way to look at this is an example: 

\begin{verbatim}
var a = 5

if (a == 5) {
    println("True")
}
\end{verbatim}

Since \verb!a! is indeed equal to 5, then the program will execute
\verb!println("True")! however, let's say we had\dots

\begin{verbatim}
var a = 5

if (a == 6) {
    println("True")
}
\end{verbatim}

Since \verb!a! doesn't equal to 6, the expression 
results to \verb!false!, so the program doesn't print anything.
\newpage
\section{Else Statements}
An \verb!else! statement is used to execute a section of code if an 
expression results to \verb!false!. Let's fix our previous example a bit to include an \verb!else! statement: 

\begin{verbatim}
var a = 5

if (a == 6) {
    println("True")
} else {
    println("False")
}
\end{verbatim}

Since \verb!a! doesn't equal to 6, it follows the \verb!else! statement, and executes \verb!println("False")!.

\section{Else If Statements}
An \verb!else if! expression allows for more conditions to be evaluated within an \verb!if/else! expression.
\\\\
\textbf{Note:} You can use multiple else if expressions as long as they appear after the if expression and before the else expression.

\begin{verbatim}
var num = 8 

if (num > 8){
    println(num)
} else if (num == 8){
    println(num++)
} else {
    println(num--)
}
// Prints 9
\end{verbatim}

\section{Nested Conditionals}
A nested conditional is a conditional statement within another conditional statement. Essentially an \verb!if! statement inside 
an \verb!if! statement. 

\begin{verbatim}
var foo = true 
var bar = true 

if (foo){
    println("FOO!")
    if (bar){
        println("BAR!")
    }
} else {
    println("No Foo or Bar for you")
}


/* 
Prints: 
FOO!
BAR!
*/
\end{verbatim}

\section{When Statements}
After a while you may get sick of a huge tree of \verb!else if! statements, so any better option? 
The \verb!when! statement is useful to execute code depending on the value of the expression. 

\begin{verbatim}
var grade = "F"

when (grade){
    "A" -> println("Excellent")
    "B" -> println("Okay")
    "C" -> println("Bad")
    "D" -> println("Terrible")
    else -> println("FAIL")
}
// Prints FAIL
\end{verbatim}

\section{Comparison Operators}
Comparison operators are symbols that are used to compare two 
values in order to return a result of \verb!true! or \verb!false!.

The comparison operators include the following: 
\begin{itemize}
    \item \verb!<! less than
    \item \verb!>! greater than
    \item \verb!<=! less than or equal to
    \item \verb!>=! greater than or equal to
\end{itemize}

\begin{verbatim}
var Khloe = 19
var Zane = 18
var Nimu = 20

Khloe > Zane //true 
Zane < Nimu // true 
Khloe >= Nimu // false 
Nimu <= Zane // false
\end{verbatim}

\section{Logic Operators}
Logical operators are symbols used to evaluate the relationship 
between two or more Boolean expressions in order to return a \verb!true! or \verb!false! value.

The logic operators include the following:
\begin{itemize}
    \item ! NOT
    \item \verb!&&! AND
    \item \verb!||! OR
\end{itemize}

\paragraph{"!" NOT Operator}
The NOT operator evaluates a boolean expression and return it's opposite value \verb!(true -> false)!. 

\begin{verbatim}
var exp = true 
println(!exp)
// Prints false
\end{verbatim}

\paragraph{"\&\&" AND Operator}
The AND operator only evaluates to \verb!true! if both expressions it's evaluating results to \verb!true!. 
\begin{verbatim}
if (4 < 5 && 6 > 3){ //Both expressions are true 
    println("True")
} else {
    println("False")
}
// Prints true 
\end{verbatim}

\paragraph{"||" OR Operator}
The OR operator only evaluates to \verb!true! if one of the expressions result to \verb!true!.
\begin{verbatim}
if (1.2 < 9.8 || 5 > 9){ // One results to true 
    println("True")
} else {
    println("False")
}
//Prints True
\end{verbatim} 

\paragraph{Order of Evaluation}
The order of evaluation when using multiple logical operators in a single Boolean expression is:
\begin{enumerate}
    \item Expressions placed in parantheses
    \item NOT(!) Operator
    \item AND(\verb!&&!) Operator
    \item OR(\verb!||!) Operator
\end{enumerate}
Ex. 
\begin{verbatim}
    !true || (true && false) // false
\end{verbatim}

\section{The Range Operator}
The range operator (\verb!..!) is used to create a succession of number or character values.
\begin{verbatim}
var grades = 90

when (grades){
    90..101 -> println("A+")
    80..90 ->  println("A")
    .
    .
    .
    else -> println("F")
}
// Prints A+ 
\end{verbatim}