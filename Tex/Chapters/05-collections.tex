\chapter{Collections}
In Kotlin, there is various type of collections that exist for different case usages. In this section, we
will be covering \verb!Lists!, \verb!Sets! and \verb!Maps!.

\section{Lists}
\subsection{Mutable and Immutable Lists}
\begin{itemize}
    \item An immutable list represents a group of elements with read-only operations.
          \begin{itemize}
              \item It can be declared with the term \verb!listOf!, followed by a pair of parentheses containing elements that are separated by commas.
              \item Ex. \verb!var fruits = listOf("Apples", "Bananas", "Oranges")!
          \end{itemize}
\end{itemize}

\begin{itemize}
    \item A mutable list represents a group of ordered elements with read and write operations.
          \begin{itemize}
              \item It can be declared with the term, \verb!mutableListOf! followed by a pair of parentheses containing elements that are separated by commas.
              \item Ex. \verb!var companies = mutableListOf("Google", "Tesla", "Apple")!
          \end{itemize}
\end{itemize}

\paragraph{Accessing Lists Elements}
In order to retrieve an element from a list, we can reference its numerical position or index using square bracket notation \verb![ ]!.

\textbf{Note: }The beginning index of a list starts at \verb!0!.

\paragraph{The Size Property}
The \verb!size! property returns the size of a collection of the number of elements that exists.

\begin{verbatim}
var grades = listOf("A", "B", "C", "D", "F")
println(grades.size)
// Prints 5
\end{verbatim}
\newpage
\subsection{List Operations}
The list collection supports various operations in the form of built-in functions that can be performed on its elements.

Some functions perform read and write operations, whereas others perform read-only operations.

The functions that perform read and write operations can only be used on \verb!mutable! lists while read-only operations can be
performed on both \verb!mutable! and \verb!immutable! lists.

\begin{verbatim}
var programming_languages = mutableListOf("Kotlin", "Java", "C++", "Rust")

if (programming_languages.contains("Python") /*read only*/){
  println("Python is pretty cool")
} else {
  programming_languages.add("Python") //write 
}
\end{verbatim}

\section{Sets}
\subsection{Immutable and Mutable Sets}
\begin{itemize}
    \item An immutable set represents a collection of unique elements in an unordered format whose elements cannot be changed throughout a program.
          \begin{itemize}
              \item It is declared with the term, \verb!setOf!, followed by a pair of parentheses,
                    \verb!( )! holding unique values.
              \item \verb!var origin_factions = setOf("Mythicals", "Sorcerers", "Kindgom")!
          \end{itemize}
\end{itemize}

\begin{itemize}
    \item A mutable set represents a collection of ordered elements that possess both read and write functionalities.
          \begin{itemize}
              \item It is declared with the term, \verb!mutableSetOf!, followed by a pair of parantheses, \verb!( )! holding unique values.
              \item \verb!var mkproj = mutableSetOf("Phaktionz", "Books", "UniConv", "Moka")!
          \end{itemize}
\end{itemize}

\paragraph{Accessing Set Elements}
Elements in a set can be accessed using the \verb!elementAt()! or \verb!elementAtOrNull()! functions.

\begin{itemize}
    \item The \verb!elementAt()! function gets appended onto a set name and returns the element at the specified position within the parentheses.
    \item The \verb!elementAtOrNull()! function is a safer variation of the \verb!elementAt()! function
          \begin{itemize}
              \item Returns \verb!null! if the position is out of bounds as opposed to throwing an error.
          \end{itemize}
\end{itemize}

\begin{verbatim}
var example = setOf("Foo", "Bar", "Baz")

println(example.elementAt(1)) // Prints Bar
println(example.elementAt(5)) // Returns an error 
println(example.elementAtOrNull(5)) // Prints null
\end{verbatim}
\newpage
\section{Maps}
\subsection{Immutable and Mutable Maps}
\begin{itemize}
    \item An immutable Map represents a collection of entries that cannot be altered throughout a program.
          \begin{itemize}
              \item It is declared with the term, \verb!mapOf!, followed by a pair of parentheses.
                    \begin{itemize}
                        \item Within the parentheses, each key should be linked to its corresponding value with the
                              \verb!to! keyword, and each entry should be separated by a comma.
                        \item \verb!var student = mapOf("Johnny" to 95, "Billy  to 65, "Kimmy" to 85)!
                    \end{itemize}
          \end{itemize}
\end{itemize}

\begin{itemize}
    \item A mutable map represents a collection of entries that possess read and write functionalities. 
    Entries can be added, removed, or updated in a mutable map.
    \begin{itemize}
        \item A mutable map can be declared with the term, \verb!mutableMapOf!, 
        followed by a pair of parentheses holding key-value pairs.
        \item \verb!var prices = mutableMapOf("Gum" to 1.50, "Kitkat" to 0.88, "Spinach" to 3.97)!
    \end{itemize}
\end{itemize}

\paragraph{Retrieving Map Keys and Values}
\begin{itemize}
    \item Keys and values within a map can be retrieved using the \verb!.keys! and \verb!.values! properties.
    \item The \verb!.keys! property returns a list of key elements, whereas the \verb!.values! property returns a list of value elements.
    \item To retrieve a single value associated with a key, the shorthand, \verb![key]!, syntax can be used.
\end{itemize}

\begin{verbatim}
var albums = mapOf("2001" to "Dr Dre", "4:44" to "Jay Z", "Relapse" to "Eminem")

println(albums.keys)
// Prints ["2001", "4:44", "Relapse"]
println(albums.values)
// Prints ["Dr Dre", "Jay Z", "Eminem" ]
println(albums["4:44"])
// Prints Jay Z
\end{verbatim}

\paragraph{Adding and Removing Map Entries}
An entry can be added to a mutable map using the \verb!put()! function. 
Oppositely, an entry can be removed from a mutable map using the \verb!remove()! function.

\begin{itemize}
    \item The \verb!put()! function accepts a key and a value separated by a comma.
    \item The \verb!remove()! function accepts a key and removes the entry associated with that key
\end{itemize}

\begin{verbatim}
var albums = mutableMapOf("2001" to "Dr Dre", "4:44" to "Jay Z", "Relapse" to "Eminem")

albums.put("All Eyez On Me", "2Pac")

println(albums)
// Prints: 
// {"2001"="Dr Dre", "4:44"="Jay Z", "Relapse"="Eminem", "All Eyez On Me"="2Pac"}

albums.remove("2001")
println(albums)
// Prints: 
// {"4:44"="Jay Z", "Relapse"="Eminem", "All Eyez On Me"="2Pac"}
\end{verbatim}