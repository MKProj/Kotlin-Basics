\chapter{Data Types}
We will begin this section off by disecting this example program that contains most topics that will be discussed: 
\begin{verbatim}
    fun main(){
        /*
         Our goal is to be able to have variables that 
         calculate various geometric areas
         */ 
    
        // An immutable variable for the value of pi
        val pi = 3.14159
    
        var radius = 10.0
        var area = pi * radius // Circle 
    
        //Since we will be using the area statement a lot
        //it's good to make it a variables 
    
        var stm = "The area of this "
        var shape = "circle"
    
        println(stm + shape + " is $area") // String Concetation + Template 
    
        // Now let's say we want to find the area and volume of a square 

        var length = 56.21
        area = Math.pow(length, 2.00) // Will square the length
        var volume = Math.pow(length, 3.00) // Will cube the length
    
        // Now to prepare this statement 
        // However I want each beginning letter of each word capatalized 
    
        shape = "square"

        stm = stm + shape + " is $area" + 
        "\nThe volume of this " + shape + " is $volume"

        println(stm.capitalize())
    
    }  
\end{verbatim}

\section{Mutable and Immutable Variables}
\par Variables are used in a program to store data, and in Kotlin there is two different ways to declare a variable, 
one being \verb!var! which is for mutable variables, and the other \verb!val! being for immutable. Now what is the difference between 
mutable and immutable, well the difference is between being able to be reassigned. A mutable variable is a variable that can be 
reassigned any amount of times in a program, while a immutable cannot be reassigned a value once it is given one.  

\par You can see this in the program in the use case of \verb!val pi = 3.14159! where we don't want \verb!pi! to ever change. You 
can see a few of our varaibles were reassigned, such as \verb!shape!, \verb!area!, etc.  

\subsection{Type Inference}
If you're coming from Java to Kotlin, you must've been being so confused by the no semi colons, or the fact that when you declare a
variable, you don't need to provide a type with it. This is what Kotlin means with type inference, where the Kotlin compiler at 
compiling time can infer what data type each varaible is.  
\\\\
A nice way to look at Kotlin's type inference is to compare Kotlin and Java: 
\\
\begin{tabular}{|c|c|}
    \hline
    Java & Kotlin \\
    \hline
    \verb!double num_double = 67.89;! & \verb!var num_double = 67.89! \\
    \hline
\end{tabular}
\\
In Java a data type must be declared explicitly while in Kotlin, the type is implicitly inferred. 

\section{Strings}
Strings are essentially an array of characters, and in Java is not considered to be a primitive type, 
however, Strings do have useful builtin properties that are good to take advantage of. 

\paragraph{String Concentation}
String Concetation is to combine Strings together with the \verb!+! operator, and as you can see, 
it was useful for formatting our statement with the various variables we had with us. 
\\\\
Ex. 
\begin{verbatim}
stm = stm + shape + " is $area" + 
"\nThe volume of this " + shape + " is $volume"
\end{verbatim}

\paragraph{String Templates}
String Templates, which is also used in our example is useful to have variables in a String with \verb!$variable!
inside the String to cleanly write the strings without needing to concetate a lot.  

\paragraph{String Builtin Properties}
In the ending of the program, we use a builtin String function called \verb!.capatilize()! which served our
purpose to capatalize each word in a String. There is also \verb!.length()! which returns the number of characters in a String. 
It is a good idea to explore the various String properties that Kotlin offers. 

\paragraph{Character Escape Sequences}
Character escape sequences consist of a backslash and character and are used to format text.

\begin{itemize}
    \item \verb!\n! Inserts a new line
    \item \verb!\t! Inserts a tab
    \item \verb!\r! Inserts a carriage return
    \item \verb!\'! Inserts a single quote
    \item \verb!\"! Inserts a double quote
    \item \verb!\\! Inserts a backslash
    \item \verb!\$! Inserts a dollar symbol
\end{itemize}

\section{Math}
\paragraph{Arithmetic Operators}
Kotlin includes the following arithmetic operators: 
\begin{itemize}
    \item \verb!+! addition
    \item \verb!-! subtraction
    \item \verb!*! multiplication
    \item \verb!/! division
    \item \verb!%! modulus
\end{itemize}

\paragraph{Order of Operations}
The order of operations for compound arithmetic expressions is as follows:

\begin{enumerate}
    \item Parantheses
    \item Multiplication
    \item Division
    \item Modulus
    \item Addition
    \item Subtraction
\end{enumerate}


When an expression contains operations such as multiplication and division or addition and 
subtraction side by side, the compiler will evaluate the expression in a left to right order.

\paragraph{Augmented Assignment Operators}
An augmented assignment operator includes a single arithmetic and assignment operator used 
to calculate and reassign a value in one step.

\begin{verbatim}
    var a = 10
    var b = 7

    a += b // 10 + 7
    a -= b // 17 - 7
    a *= b // 10 * 7
    a /= b // 70 / 7
    a %= b // 10 % 7
\end{verbatim}

\paragraph{Increment and Decrement Operators}
Increment and decrement operators provide a shorthand syntax for adding or subtracting 1 from a value. 
An increment operator consists of two consecutive plus symbols, \verb!++!, meanwhile a decrement operator 
consists of two consecutive minus symbols, \verb!--!.

\begin{verbatim}
    var num = 78

    num++ // 79
    num-- // 78
\end{verbatim}

\textit{The Increment Operator is commonly used in loops (Chapter 6)}

\paragraph{The Math Library}
The Math library, inherited from Java, contains various mathematical 
functions that can be used within a Kotlin program.

\begin{verbatim}
    Math.pow(2.0, 3.0)  // 8.0
    Math.min(6, 9)      // 6 
    Math.max(10, 12)    // 12
    Math.round(13.7)    // 14
\end{verbatim}