\chapter{Loops}
\section{for Loops}
A \verb!for! loop is used when we know how many times we want a section of code repeated. 

Let's examine the following code: 

\begin{verbatim}
var list = listOf["Hello", "Example", "Test"]

for (i in 0..list){
    println(list[i])
}
/* Prints: 
Hello
Example
Test 
*/
\end{verbatim}

\begin{itemize}
    \item \verb!for! is a keyword used to declare a for loop.
    \item We define \verb!i! as the loop variable. 
    This variable holds the current iteration value and can be used within the loop body.
    \item The \verb!in! keyword is between the variable definition and the iterator.
    \item The range \verb!0..list! is the for loop iterator.
\end{itemize}

An \verb!iterator! is an object that allows us to step through and access every individual element in a collection of values. 

\textbf{Note: } It is important to note that the loop variable only exists within the loop’s code block. 
Trying to access the loop variable outside the \verb!for! loop will result in an error.
\newpage
\subsection{Controlling Iteration}
Sometimes we want to count backwards, or count by 5s, or maybe both! Using certain functions alongside or instead of the normal 
range operator \verb!(..)! can enhance the iterative abilities of our \verb!for! loops. The functions \verb!downTo!, \verb!until! and \verb!step! give us 
more control of a range and therefore more control of our loops.

\begin{itemize}
    \item The \verb!downTo! function simply creates a reverse order group of values, where the starting boundary is greater than the ending boundary.
To accomplish this, replace the range operator \verb!(..)! with \verb!downTo!:
\end{itemize}

\begin{verbatim}
for (i in 4 downTo 1) {
  println("i = $i")
}
/*
Output: 
i = 4
i = 3
i = 2
i = 1
*/
\end{verbatim}
We can see in the output that the first number in \verb!i! is 4 and the last is 1  

\begin{itemize}
    \item The \verb!until! function creates an ascending range, just like the 
    \verb!(..)! operator, but excludes the upper boundary:
\end{itemize}

\begin{verbatim}
for (i in 1 until 4) {
  println("i = $i")
}
/*
Output: 
i = 1
i = 2
i = 3
*/
\end{verbatim}
The upper boundary, 4, is not included in the output 

\begin{itemize}
    \item Up until now, each of these functions, including the range operator (..), have counted up or down by one. 
    The \verb!step! function specifies the amount these functions count by:
\end{itemize}

\begin{verbatim}
for (i in 1..8 step 2) {
  println("i = $i")
}
/*
Output: 
i = 1
i = 3
i = 5
i = 7
*/
\end{verbatim}

The loop variable \verb!i! now increases by 2 for every iteration. The last number output is \verb!7!, 
since 2 steps up from that is \verb!9! which is outside the defined range, \verb!1..8!